% Options for packages loaded elsewhere
\PassOptionsToPackage{unicode}{hyperref}
\PassOptionsToPackage{hyphens}{url}
%
\documentclass[
]{article}
\usepackage{amsmath,amssymb}
\usepackage{iftex}
\ifPDFTeX
  \usepackage[T1]{fontenc}
  \usepackage[utf8]{inputenc}
  \usepackage{textcomp} % provide euro and other symbols
\else % if luatex or xetex
  \usepackage{unicode-math} % this also loads fontspec
  \defaultfontfeatures{Scale=MatchLowercase}
  \defaultfontfeatures[\rmfamily]{Ligatures=TeX,Scale=1}
\fi
\usepackage{lmodern}
\ifPDFTeX\else
  % xetex/luatex font selection
\fi
% Use upquote if available, for straight quotes in verbatim environments
\IfFileExists{upquote.sty}{\usepackage{upquote}}{}
\IfFileExists{microtype.sty}{% use microtype if available
  \usepackage[]{microtype}
  \UseMicrotypeSet[protrusion]{basicmath} % disable protrusion for tt fonts
}{}
\makeatletter
\@ifundefined{KOMAClassName}{% if non-KOMA class
  \IfFileExists{parskip.sty}{%
    \usepackage{parskip}
  }{% else
    \setlength{\parindent}{0pt}
    \setlength{\parskip}{6pt plus 2pt minus 1pt}}
}{% if KOMA class
  \KOMAoptions{parskip=half}}
\makeatother
\usepackage{xcolor}
\usepackage[margin=1in]{geometry}
\usepackage{color}
\usepackage{fancyvrb}
\newcommand{\VerbBar}{|}
\newcommand{\VERB}{\Verb[commandchars=\\\{\}]}
\DefineVerbatimEnvironment{Highlighting}{Verbatim}{commandchars=\\\{\}}
% Add ',fontsize=\small' for more characters per line
\usepackage{framed}
\definecolor{shadecolor}{RGB}{248,248,248}
\newenvironment{Shaded}{\begin{snugshade}}{\end{snugshade}}
\newcommand{\AlertTok}[1]{\textcolor[rgb]{0.94,0.16,0.16}{#1}}
\newcommand{\AnnotationTok}[1]{\textcolor[rgb]{0.56,0.35,0.01}{\textbf{\textit{#1}}}}
\newcommand{\AttributeTok}[1]{\textcolor[rgb]{0.13,0.29,0.53}{#1}}
\newcommand{\BaseNTok}[1]{\textcolor[rgb]{0.00,0.00,0.81}{#1}}
\newcommand{\BuiltInTok}[1]{#1}
\newcommand{\CharTok}[1]{\textcolor[rgb]{0.31,0.60,0.02}{#1}}
\newcommand{\CommentTok}[1]{\textcolor[rgb]{0.56,0.35,0.01}{\textit{#1}}}
\newcommand{\CommentVarTok}[1]{\textcolor[rgb]{0.56,0.35,0.01}{\textbf{\textit{#1}}}}
\newcommand{\ConstantTok}[1]{\textcolor[rgb]{0.56,0.35,0.01}{#1}}
\newcommand{\ControlFlowTok}[1]{\textcolor[rgb]{0.13,0.29,0.53}{\textbf{#1}}}
\newcommand{\DataTypeTok}[1]{\textcolor[rgb]{0.13,0.29,0.53}{#1}}
\newcommand{\DecValTok}[1]{\textcolor[rgb]{0.00,0.00,0.81}{#1}}
\newcommand{\DocumentationTok}[1]{\textcolor[rgb]{0.56,0.35,0.01}{\textbf{\textit{#1}}}}
\newcommand{\ErrorTok}[1]{\textcolor[rgb]{0.64,0.00,0.00}{\textbf{#1}}}
\newcommand{\ExtensionTok}[1]{#1}
\newcommand{\FloatTok}[1]{\textcolor[rgb]{0.00,0.00,0.81}{#1}}
\newcommand{\FunctionTok}[1]{\textcolor[rgb]{0.13,0.29,0.53}{\textbf{#1}}}
\newcommand{\ImportTok}[1]{#1}
\newcommand{\InformationTok}[1]{\textcolor[rgb]{0.56,0.35,0.01}{\textbf{\textit{#1}}}}
\newcommand{\KeywordTok}[1]{\textcolor[rgb]{0.13,0.29,0.53}{\textbf{#1}}}
\newcommand{\NormalTok}[1]{#1}
\newcommand{\OperatorTok}[1]{\textcolor[rgb]{0.81,0.36,0.00}{\textbf{#1}}}
\newcommand{\OtherTok}[1]{\textcolor[rgb]{0.56,0.35,0.01}{#1}}
\newcommand{\PreprocessorTok}[1]{\textcolor[rgb]{0.56,0.35,0.01}{\textit{#1}}}
\newcommand{\RegionMarkerTok}[1]{#1}
\newcommand{\SpecialCharTok}[1]{\textcolor[rgb]{0.81,0.36,0.00}{\textbf{#1}}}
\newcommand{\SpecialStringTok}[1]{\textcolor[rgb]{0.31,0.60,0.02}{#1}}
\newcommand{\StringTok}[1]{\textcolor[rgb]{0.31,0.60,0.02}{#1}}
\newcommand{\VariableTok}[1]{\textcolor[rgb]{0.00,0.00,0.00}{#1}}
\newcommand{\VerbatimStringTok}[1]{\textcolor[rgb]{0.31,0.60,0.02}{#1}}
\newcommand{\WarningTok}[1]{\textcolor[rgb]{0.56,0.35,0.01}{\textbf{\textit{#1}}}}
\usepackage{graphicx}
\makeatletter
\def\maxwidth{\ifdim\Gin@nat@width>\linewidth\linewidth\else\Gin@nat@width\fi}
\def\maxheight{\ifdim\Gin@nat@height>\textheight\textheight\else\Gin@nat@height\fi}
\makeatother
% Scale images if necessary, so that they will not overflow the page
% margins by default, and it is still possible to overwrite the defaults
% using explicit options in \includegraphics[width, height, ...]{}
\setkeys{Gin}{width=\maxwidth,height=\maxheight,keepaspectratio}
% Set default figure placement to htbp
\makeatletter
\def\fps@figure{htbp}
\makeatother
\setlength{\emergencystretch}{3em} % prevent overfull lines
\providecommand{\tightlist}{%
  \setlength{\itemsep}{0pt}\setlength{\parskip}{0pt}}
\setcounter{secnumdepth}{-\maxdimen} % remove section numbering
\ifLuaTeX
  \usepackage{selnolig}  % disable illegal ligatures
\fi
\IfFileExists{bookmark.sty}{\usepackage{bookmark}}{\usepackage{hyperref}}
\IfFileExists{xurl.sty}{\usepackage{xurl}}{} % add URL line breaks if available
\urlstyle{same}
\hypersetup{
  pdftitle={Analysis of Divvy Cycle Data},
  pdfauthor={Pallavi},
  hidelinks,
  pdfcreator={LaTeX via pandoc}}

\title{Analysis of Divvy Cycle Data}
\author{Pallavi}
\date{2024-01-29}

\begin{document}
\maketitle

\hypertarget{introduction-to-problem}{%
\subsubsection{Introduction to Problem:}\label{introduction-to-problem}}

A bike-share company in Chicago features more than 5,800 bicycles and
600 docking stations. Cyclist sets itself apart by also offering
reclining bikes, hand tricycles, and cargo bikes, making bike-share more
inclusive to people with disabilities and riders who can't use a
standard two-wheeled bike. The majority of riders opt for traditional
bikes; about 8\% of riders use the assistive options. Users are this
program are categorized into two ways: Members, who owns the annual
membership and the casual riders, who don't own the membership but pays
hourly based on the time they use the bike.\\

The director of marketing team at cyclist, believes the company's future
success depends on maximizing the number of annual memberships.
Therefore, I want to understand how casual riders and annual members use
Cyclist bikes differently. From these insights, my team will design a
new marketing strategy to convert casual riders into annual members.\\

\hypertarget{approach}{%
\subsubsection{Approach}\label{approach}}

To get the results, I followed all the phases of data analysis which are
ASK, PREPARE, PROCESS, ANALYSE, SHARE, ACT.\\

\hypertarget{ask}{%
\subsubsection{ASK}\label{ask}}

To find out how annual members and casual riders use Cyclist bikes
differently. This insights would help the team to design a marketing
strategy targeted at converting casual riders to members to promote the
growth of Cyclist.

\hypertarget{prepare}{%
\subsubsection{PREPARE}\label{prepare}}

The data has been made available by Motivate International Inc.~under
\href{https://divvybikes.com/data-license-agreement}{this license}. I
will be using the trip data from January 2021 to December 2021. \#\#\#
PROCESS I have used R to analyze the data. I chose R programming
language because of its flexibility in data manipulation and
visualization. Following is the list of codes that I performed for
solving this problem.

\hypertarget{loading-the-required-libraries}{%
\subsection{Loading the required
libraries}\label{loading-the-required-libraries}}

\begin{Shaded}
\begin{Highlighting}[]
\FunctionTok{library}\NormalTok{(tidyverse)}
\end{Highlighting}
\end{Shaded}

\begin{verbatim}
## -- Attaching core tidyverse packages ------------------------ tidyverse 2.0.0 --
## v dplyr     1.1.4     v readr     2.1.5
## v forcats   1.0.0     v stringr   1.5.1
## v ggplot2   3.4.4     v tibble    3.2.1
## v lubridate 1.9.3     v tidyr     1.3.0
## v purrr     1.0.2     
## -- Conflicts ------------------------------------------ tidyverse_conflicts() --
## x dplyr::filter() masks stats::filter()
## x dplyr::lag()    masks stats::lag()
## i Use the conflicted package (<http://conflicted.r-lib.org/>) to force all conflicts to become errors
\end{verbatim}

\begin{Shaded}
\begin{Highlighting}[]
\FunctionTok{library}\NormalTok{(lubridate)}
\FunctionTok{library}\NormalTok{(janitor)}
\end{Highlighting}
\end{Shaded}

\begin{verbatim}
## 
## Attaching package: 'janitor'
## 
## The following objects are masked from 'package:stats':
## 
##     chisq.test, fisher.test
\end{verbatim}

\begin{Shaded}
\begin{Highlighting}[]
\FunctionTok{library}\NormalTok{(readr)}
\end{Highlighting}
\end{Shaded}

\hypertarget{loading-the-2021-trip-data}{%
\subsection{Loading the 2021 trip
data}\label{loading-the-2021-trip-data}}

\begin{Shaded}
\begin{Highlighting}[]
\NormalTok{jan }\OtherTok{\textless{}{-}} \FunctionTok{read\_csv}\NormalTok{(}\StringTok{"jan.csv"}\NormalTok{)}
\end{Highlighting}
\end{Shaded}

\begin{verbatim}
## Rows: 103770 Columns: 13
## -- Column specification --------------------------------------------------------
## Delimiter: ","
## chr  (7): ride_id, rideable_type, start_station_name, start_station_id, end_...
## dbl  (4): start_lat, start_lng, end_lat, end_lng
## dttm (2): started_at, ended_at
## 
## i Use `spec()` to retrieve the full column specification for this data.
## i Specify the column types or set `show_col_types = FALSE` to quiet this message.
\end{verbatim}

\begin{Shaded}
\begin{Highlighting}[]
\NormalTok{feb }\OtherTok{\textless{}{-}} \FunctionTok{read\_csv}\NormalTok{(}\StringTok{"feb.csv"}\NormalTok{)}
\end{Highlighting}
\end{Shaded}

\begin{verbatim}
## Rows: 115609 Columns: 13
## -- Column specification --------------------------------------------------------
## Delimiter: ","
## chr  (7): ride_id, rideable_type, start_station_name, start_station_id, end_...
## dbl  (4): start_lat, start_lng, end_lat, end_lng
## dttm (2): started_at, ended_at
## 
## i Use `spec()` to retrieve the full column specification for this data.
## i Specify the column types or set `show_col_types = FALSE` to quiet this message.
\end{verbatim}

\begin{Shaded}
\begin{Highlighting}[]
\NormalTok{mar }\OtherTok{\textless{}{-}} \FunctionTok{read\_csv}\NormalTok{(}\StringTok{"mar.csv"}\NormalTok{)}
\end{Highlighting}
\end{Shaded}

\begin{verbatim}
## Rows: 284042 Columns: 13
## -- Column specification --------------------------------------------------------
## Delimiter: ","
## chr  (7): ride_id, rideable_type, start_station_name, start_station_id, end_...
## dbl  (4): start_lat, start_lng, end_lat, end_lng
## dttm (2): started_at, ended_at
## 
## i Use `spec()` to retrieve the full column specification for this data.
## i Specify the column types or set `show_col_types = FALSE` to quiet this message.
\end{verbatim}

\begin{Shaded}
\begin{Highlighting}[]
\NormalTok{april }\OtherTok{\textless{}{-}} \FunctionTok{read\_csv}\NormalTok{(}\StringTok{"april.csv"}\NormalTok{)}
\end{Highlighting}
\end{Shaded}

\begin{verbatim}
## Rows: 371249 Columns: 13
## -- Column specification --------------------------------------------------------
## Delimiter: ","
## chr  (7): ride_id, rideable_type, start_station_name, start_station_id, end_...
## dbl  (4): start_lat, start_lng, end_lat, end_lng
## dttm (2): started_at, ended_at
## 
## i Use `spec()` to retrieve the full column specification for this data.
## i Specify the column types or set `show_col_types = FALSE` to quiet this message.
\end{verbatim}

\begin{Shaded}
\begin{Highlighting}[]
\NormalTok{may }\OtherTok{\textless{}{-}} \FunctionTok{read\_csv}\NormalTok{(}\StringTok{"may.csv"}\NormalTok{)}
\end{Highlighting}
\end{Shaded}

\begin{verbatim}
## Rows: 634858 Columns: 13
## -- Column specification --------------------------------------------------------
## Delimiter: ","
## chr  (7): ride_id, rideable_type, start_station_name, start_station_id, end_...
## dbl  (4): start_lat, start_lng, end_lat, end_lng
## dttm (2): started_at, ended_at
## 
## i Use `spec()` to retrieve the full column specification for this data.
## i Specify the column types or set `show_col_types = FALSE` to quiet this message.
\end{verbatim}

\begin{Shaded}
\begin{Highlighting}[]
\NormalTok{june }\OtherTok{\textless{}{-}} \FunctionTok{read\_csv}\NormalTok{(}\StringTok{"june.csv"}\NormalTok{)}
\end{Highlighting}
\end{Shaded}

\begin{verbatim}
## Rows: 769204 Columns: 13
## -- Column specification --------------------------------------------------------
## Delimiter: ","
## chr  (7): ride_id, rideable_type, start_station_name, start_station_id, end_...
## dbl  (4): start_lat, start_lng, end_lat, end_lng
## dttm (2): started_at, ended_at
## 
## i Use `spec()` to retrieve the full column specification for this data.
## i Specify the column types or set `show_col_types = FALSE` to quiet this message.
\end{verbatim}

\begin{Shaded}
\begin{Highlighting}[]
\NormalTok{july }\OtherTok{\textless{}{-}} \FunctionTok{read\_csv}\NormalTok{(}\StringTok{"july.csv"}\NormalTok{)}
\end{Highlighting}
\end{Shaded}

\begin{verbatim}
## Rows: 823488 Columns: 13
## -- Column specification --------------------------------------------------------
## Delimiter: ","
## chr  (7): ride_id, rideable_type, start_station_name, start_station_id, end_...
## dbl  (4): start_lat, start_lng, end_lat, end_lng
## dttm (2): started_at, ended_at
## 
## i Use `spec()` to retrieve the full column specification for this data.
## i Specify the column types or set `show_col_types = FALSE` to quiet this message.
\end{verbatim}

\begin{Shaded}
\begin{Highlighting}[]
\NormalTok{august }\OtherTok{\textless{}{-}} \FunctionTok{read\_csv}\NormalTok{(}\StringTok{"august.csv"}\NormalTok{)}
\end{Highlighting}
\end{Shaded}

\begin{verbatim}
## Rows: 785932 Columns: 13
## -- Column specification --------------------------------------------------------
## Delimiter: ","
## chr  (7): ride_id, rideable_type, start_station_name, start_station_id, end_...
## dbl  (4): start_lat, start_lng, end_lat, end_lng
## dttm (2): started_at, ended_at
## 
## i Use `spec()` to retrieve the full column specification for this data.
## i Specify the column types or set `show_col_types = FALSE` to quiet this message.
\end{verbatim}

\begin{Shaded}
\begin{Highlighting}[]
\NormalTok{september }\OtherTok{\textless{}{-}} \FunctionTok{read\_csv}\NormalTok{(}\StringTok{"september.csv"}\NormalTok{)}
\end{Highlighting}
\end{Shaded}

\begin{verbatim}
## Rows: 701339 Columns: 13
## -- Column specification --------------------------------------------------------
## Delimiter: ","
## chr  (7): ride_id, rideable_type, start_station_name, start_station_id, end_...
## dbl  (4): start_lat, start_lng, end_lat, end_lng
## dttm (2): started_at, ended_at
## 
## i Use `spec()` to retrieve the full column specification for this data.
## i Specify the column types or set `show_col_types = FALSE` to quiet this message.
\end{verbatim}

\begin{Shaded}
\begin{Highlighting}[]
\NormalTok{october }\OtherTok{\textless{}{-}} \FunctionTok{read\_csv}\NormalTok{(}\StringTok{"october.csv"}\NormalTok{)}
\end{Highlighting}
\end{Shaded}

\begin{verbatim}
## Rows: 558685 Columns: 13
## -- Column specification --------------------------------------------------------
## Delimiter: ","
## chr  (7): ride_id, rideable_type, start_station_name, start_station_id, end_...
## dbl  (4): start_lat, start_lng, end_lat, end_lng
## dttm (2): started_at, ended_at
## 
## i Use `spec()` to retrieve the full column specification for this data.
## i Specify the column types or set `show_col_types = FALSE` to quiet this message.
\end{verbatim}

\begin{Shaded}
\begin{Highlighting}[]
\NormalTok{november }\OtherTok{\textless{}{-}} \FunctionTok{read\_csv}\NormalTok{(}\StringTok{"november.csv"}\NormalTok{)}
\end{Highlighting}
\end{Shaded}

\begin{verbatim}
## Rows: 337735 Columns: 13
## -- Column specification --------------------------------------------------------
## Delimiter: ","
## chr  (7): ride_id, rideable_type, start_station_name, start_station_id, end_...
## dbl  (4): start_lat, start_lng, end_lat, end_lng
## dttm (2): started_at, ended_at
## 
## i Use `spec()` to retrieve the full column specification for this data.
## i Specify the column types or set `show_col_types = FALSE` to quiet this message.
\end{verbatim}

\begin{Shaded}
\begin{Highlighting}[]
\NormalTok{december }\OtherTok{\textless{}{-}} \FunctionTok{read\_csv}\NormalTok{(}\StringTok{"december.csv"}\NormalTok{)}
\end{Highlighting}
\end{Shaded}

\begin{verbatim}
## Rows: 181806 Columns: 13
## -- Column specification --------------------------------------------------------
## Delimiter: ","
## chr  (7): ride_id, rideable_type, start_station_name, start_station_id, end_...
## dbl  (4): start_lat, start_lng, end_lat, end_lng
## dttm (2): started_at, ended_at
## 
## i Use `spec()` to retrieve the full column specification for this data.
## i Specify the column types or set `show_col_types = FALSE` to quiet this message.
\end{verbatim}

\hypertarget{explore-data}{%
\subsection{Explore data}\label{explore-data}}

\begin{Shaded}
\begin{Highlighting}[]
\FunctionTok{compare\_df\_cols\_same}\NormalTok{(jan,feb,mar,april,may,june,july,august,september,october,}
\NormalTok{                     november,december)}
\end{Highlighting}
\end{Shaded}

\begin{verbatim}
## [1] TRUE
\end{verbatim}

\hypertarget{combine-all-months-data-to-a-single-file}{%
\subsection{Combine all months data to a single
file}\label{combine-all-months-data-to-a-single-file}}

\begin{Shaded}
\begin{Highlighting}[]
\NormalTok{trips2021 }\OtherTok{\textless{}{-}} \FunctionTok{rbind}\NormalTok{(jan,feb,mar,april,may,june,july,august,september,october,}
\NormalTok{                   november,december)}
\end{Highlighting}
\end{Shaded}

\hypertarget{visualise-the-data}{%
\subsection{Visualise the data}\label{visualise-the-data}}

\begin{Shaded}
\begin{Highlighting}[]
\FunctionTok{head}\NormalTok{(trips2021)}
\end{Highlighting}
\end{Shaded}

\begin{verbatim}
## # A tibble: 6 x 13
##   ride_id          rideable_type started_at          ended_at           
##   <chr>            <chr>         <dttm>              <dttm>             
## 1 C2F7DD78E82EC875 electric_bike 2022-01-13 11:59:47 2022-01-13 12:02:44
## 2 A6CF8980A652D272 electric_bike 2022-01-10 08:41:56 2022-01-10 08:46:17
## 3 BD0F91DFF741C66D classic_bike  2022-01-25 04:53:40 2022-01-25 04:58:01
## 4 CBB80ED419105406 classic_bike  2022-01-04 00:18:04 2022-01-04 00:33:00
## 5 DDC963BFDDA51EEA classic_bike  2022-01-20 01:31:10 2022-01-20 01:37:12
## 6 A39C6F6CC0586C0B classic_bike  2022-01-11 18:48:09 2022-01-11 18:51:31
## # i 9 more variables: start_station_name <chr>, start_station_id <chr>,
## #   end_station_name <chr>, end_station_id <chr>, start_lat <dbl>,
## #   start_lng <dbl>, end_lat <dbl>, end_lng <dbl>, member_casual <chr>
\end{verbatim}

\begin{Shaded}
\begin{Highlighting}[]
\FunctionTok{summary}\NormalTok{(trips2021)}
\end{Highlighting}
\end{Shaded}

\begin{verbatim}
##    ride_id          rideable_type        started_at                    
##  Length:5667717     Length:5667717     Min.   :2022-01-01 00:00:05.00  
##  Class :character   Class :character   1st Qu.:2022-05-28 19:21:05.00  
##  Mode  :character   Mode  :character   Median :2022-07-22 15:03:59.00  
##                                        Mean   :2022-07-20 07:21:18.74  
##                                        3rd Qu.:2022-09-16 07:21:29.00  
##                                        Max.   :2022-12-31 23:59:26.00  
##                                                                        
##     ended_at                      start_station_name start_station_id  
##  Min.   :2022-01-01 00:01:48.00   Length:5667717     Length:5667717    
##  1st Qu.:2022-05-28 19:43:07.00   Class :character   Class :character  
##  Median :2022-07-22 15:24:44.00   Mode  :character   Mode  :character  
##  Mean   :2022-07-20 07:40:45.33                                        
##  3rd Qu.:2022-09-16 07:39:03.00                                        
##  Max.   :2023-01-02 04:56:45.00                                        
##                                                                        
##  end_station_name   end_station_id       start_lat       start_lng     
##  Length:5667717     Length:5667717     Min.   :41.64   Min.   :-87.84  
##  Class :character   Class :character   1st Qu.:41.88   1st Qu.:-87.66  
##  Mode  :character   Mode  :character   Median :41.90   Median :-87.64  
##                                        Mean   :41.90   Mean   :-87.65  
##                                        3rd Qu.:41.93   3rd Qu.:-87.63  
##                                        Max.   :45.64   Max.   :-73.80  
##                                                                        
##     end_lat         end_lng       member_casual     
##  Min.   : 0.00   Min.   :-88.14   Length:5667717    
##  1st Qu.:41.88   1st Qu.:-87.66   Class :character  
##  Median :41.90   Median :-87.64   Mode  :character  
##  Mean   :41.90   Mean   :-87.65                     
##  3rd Qu.:41.93   3rd Qu.:-87.63                     
##  Max.   :42.37   Max.   :  0.00                     
##  NA's   :5858    NA's   :5858
\end{verbatim}

\begin{Shaded}
\begin{Highlighting}[]
\FunctionTok{glimpse}\NormalTok{(trips2021)}
\end{Highlighting}
\end{Shaded}

\begin{verbatim}
## Rows: 5,667,717
## Columns: 13
## $ ride_id            <chr> "C2F7DD78E82EC875", "A6CF8980A652D272", "BD0F91DFF7~
## $ rideable_type      <chr> "electric_bike", "electric_bike", "classic_bike", "~
## $ started_at         <dttm> 2022-01-13 11:59:47, 2022-01-10 08:41:56, 2022-01-~
## $ ended_at           <dttm> 2022-01-13 12:02:44, 2022-01-10 08:46:17, 2022-01-~
## $ start_station_name <chr> "Glenwood Ave & Touhy Ave", "Glenwood Ave & Touhy A~
## $ start_station_id   <chr> "525", "525", "TA1306000016", "KA1504000151", "TA13~
## $ end_station_name   <chr> "Clark St & Touhy Ave", "Clark St & Touhy Ave", "Gr~
## $ end_station_id     <chr> "RP-007", "RP-007", "TA1307000001", "TA1309000021",~
## $ start_lat          <dbl> 42.01280, 42.01276, 41.92560, 41.98359, 41.87785, 4~
## $ start_lng          <dbl> -87.66591, -87.66597, -87.65371, -87.66915, -87.624~
## $ end_lat            <dbl> 42.01256, 42.01256, 41.92533, 41.96151, 41.88462, 4~
## $ end_lng            <dbl> -87.67437, -87.67437, -87.66580, -87.67139, -87.627~
## $ member_casual      <chr> "casual", "casual", "member", "casual", "member", "~
\end{verbatim}

\hypertarget{checking-for-duplicate-rows}{%
\subsection{Checking for duplicate
rows}\label{checking-for-duplicate-rows}}

\begin{Shaded}
\begin{Highlighting}[]
\FunctionTok{nrow}\NormalTok{(}\FunctionTok{distinct}\NormalTok{(trips2021)) }\SpecialCharTok{==} \FunctionTok{nrow}\NormalTok{(trips2021)}
\end{Highlighting}
\end{Shaded}

\begin{verbatim}
## [1] TRUE
\end{verbatim}

\hypertarget{remove-nulls}{%
\subsection{Remove Nulls}\label{remove-nulls}}

\begin{Shaded}
\begin{Highlighting}[]
\NormalTok{trips2021\_v2 }\OtherTok{\textless{}{-}} \FunctionTok{drop\_na}\NormalTok{(trips2021)}
\FunctionTok{glimpse}\NormalTok{(trips2021\_v2)}
\end{Highlighting}
\end{Shaded}

\begin{verbatim}
## Rows: 4,369,360
## Columns: 13
## $ ride_id            <chr> "C2F7DD78E82EC875", "A6CF8980A652D272", "BD0F91DFF7~
## $ rideable_type      <chr> "electric_bike", "electric_bike", "classic_bike", "~
## $ started_at         <dttm> 2022-01-13 11:59:47, 2022-01-10 08:41:56, 2022-01-~
## $ ended_at           <dttm> 2022-01-13 12:02:44, 2022-01-10 08:46:17, 2022-01-~
## $ start_station_name <chr> "Glenwood Ave & Touhy Ave", "Glenwood Ave & Touhy A~
## $ start_station_id   <chr> "525", "525", "TA1306000016", "KA1504000151", "TA13~
## $ end_station_name   <chr> "Clark St & Touhy Ave", "Clark St & Touhy Ave", "Gr~
## $ end_station_id     <chr> "RP-007", "RP-007", "TA1307000001", "TA1309000021",~
## $ start_lat          <dbl> 42.01280, 42.01276, 41.92560, 41.98359, 41.87785, 4~
## $ start_lng          <dbl> -87.66591, -87.66597, -87.65371, -87.66915, -87.624~
## $ end_lat            <dbl> 42.01256, 42.01256, 41.92533, 41.96151, 41.88462, 4~
## $ end_lng            <dbl> -87.67437, -87.67437, -87.66580, -87.67139, -87.627~
## $ member_casual      <chr> "casual", "casual", "member", "casual", "member", "~
\end{verbatim}

\hypertarget{renaming-some-columns}{%
\subsection{Renaming some columns}\label{renaming-some-columns}}

\begin{Shaded}
\begin{Highlighting}[]
\NormalTok{trips2021\_v2 }\OtherTok{\textless{}{-}}\NormalTok{ trips2021\_v2 }\SpecialCharTok{\%\textgreater{}\%} 
  \FunctionTok{mutate}\NormalTok{(}\AttributeTok{ride\_duration =} \FunctionTok{round}\NormalTok{(}\FunctionTok{difftime}\NormalTok{(ended\_at, started\_at, }\AttributeTok{units =} \StringTok{"mins"}\NormalTok{))) }\SpecialCharTok{\%\textgreater{}\%} 
  \FunctionTok{mutate}\NormalTok{(}\AttributeTok{month =} \FunctionTok{month}\NormalTok{(started\_at, }\AttributeTok{label =} \ConstantTok{TRUE}\NormalTok{)) }\SpecialCharTok{\%\textgreater{}\%} 
  \FunctionTok{mutate}\NormalTok{(}\AttributeTok{weekday =} \FunctionTok{wday}\NormalTok{(started\_at, }\AttributeTok{label =} \ConstantTok{TRUE}\NormalTok{)) }\SpecialCharTok{\%\textgreater{}\%} 
  \FunctionTok{mutate}\NormalTok{(}\AttributeTok{start\_hour =} \FunctionTok{hour}\NormalTok{(started\_at)) }\SpecialCharTok{\%\textgreater{}\%} 
  \FunctionTok{mutate}\NormalTok{(}\AttributeTok{route =} \FunctionTok{str\_c}\NormalTok{(start\_station\_name, end\_station\_name, }\AttributeTok{sep =} \StringTok{" {-}to{-} "}\NormalTok{))}
\FunctionTok{glimpse}\NormalTok{(trips2021\_v2)}
\end{Highlighting}
\end{Shaded}

\begin{verbatim}
## Rows: 4,369,360
## Columns: 18
## $ ride_id            <chr> "C2F7DD78E82EC875", "A6CF8980A652D272", "BD0F91DFF7~
## $ rideable_type      <chr> "electric_bike", "electric_bike", "classic_bike", "~
## $ started_at         <dttm> 2022-01-13 11:59:47, 2022-01-10 08:41:56, 2022-01-~
## $ ended_at           <dttm> 2022-01-13 12:02:44, 2022-01-10 08:46:17, 2022-01-~
## $ start_station_name <chr> "Glenwood Ave & Touhy Ave", "Glenwood Ave & Touhy A~
## $ start_station_id   <chr> "525", "525", "TA1306000016", "KA1504000151", "TA13~
## $ end_station_name   <chr> "Clark St & Touhy Ave", "Clark St & Touhy Ave", "Gr~
## $ end_station_id     <chr> "RP-007", "RP-007", "TA1307000001", "TA1309000021",~
## $ start_lat          <dbl> 42.01280, 42.01276, 41.92560, 41.98359, 41.87785, 4~
## $ start_lng          <dbl> -87.66591, -87.66597, -87.65371, -87.66915, -87.624~
## $ end_lat            <dbl> 42.01256, 42.01256, 41.92533, 41.96151, 41.88462, 4~
## $ end_lng            <dbl> -87.67437, -87.67437, -87.66580, -87.67139, -87.627~
## $ member_casual      <chr> "casual", "casual", "member", "casual", "member", "~
## $ ride_duration      <drtn> 3 mins, 4 mins, 4 mins, 15 mins, 6 mins, 3 mins, 1~
## $ month              <ord> Jan, Jan, Jan, Jan, Jan, Jan, Jan, Jan, Jan, Jan, J~
## $ weekday            <ord> Thu, Mon, Tue, Tue, Thu, Tue, Sun, Sat, Mon, Fri, T~
## $ start_hour         <int> 11, 8, 4, 0, 1, 18, 18, 12, 7, 15, 18, 12, 17, 22, ~
## $ route              <chr> "Glenwood Ave & Touhy Ave -to- Clark St & Touhy Ave~
\end{verbatim}

\begin{Shaded}
\begin{Highlighting}[]
\FunctionTok{tail}\NormalTok{(trips2021\_v2, }\DecValTok{5}\NormalTok{)}
\end{Highlighting}
\end{Shaded}

\begin{verbatim}
## # A tibble: 5 x 18
##   ride_id          rideable_type started_at          ended_at           
##   <chr>            <chr>         <dttm>              <dttm>             
## 1 43ABEE85B6E15DCA classic_bike  2022-12-05 06:51:04 2022-12-05 06:54:48
## 2 F041C89A3D1F0270 electric_bike 2022-12-14 17:06:28 2022-12-14 17:19:27
## 3 A2BECB88430BE156 classic_bike  2022-12-08 16:27:47 2022-12-08 16:32:20
## 4 37B392960E566F58 classic_bike  2022-12-28 09:37:38 2022-12-28 09:41:34
## 5 2DD1587210BA45AE classic_bike  2022-12-09 00:27:25 2022-12-09 00:35:28
## # i 14 more variables: start_station_name <chr>, start_station_id <chr>,
## #   end_station_name <chr>, end_station_id <chr>, start_lat <dbl>,
## #   start_lng <dbl>, end_lat <dbl>, end_lng <dbl>, member_casual <chr>,
## #   ride_duration <drtn>, month <ord>, weekday <ord>, start_hour <int>,
## #   route <chr>
\end{verbatim}

\begin{Shaded}
\begin{Highlighting}[]
\FunctionTok{summary}\NormalTok{(trips2021\_v2)}
\end{Highlighting}
\end{Shaded}

\begin{verbatim}
##    ride_id          rideable_type        started_at                    
##  Length:4369360     Length:4369360     Min.   :2022-01-01 00:00:05.00  
##  Class :character   Class :character   1st Qu.:2022-05-29 10:30:23.00  
##  Mode  :character   Mode  :character   Median :2022-07-20 21:24:09.00  
##                                        Mean   :2022-07-19 14:07:47.16  
##                                        3rd Qu.:2022-09-14 18:22:51.50  
##                                        Max.   :2022-12-31 23:59:26.00  
##                                                                        
##     ended_at                      start_station_name start_station_id  
##  Min.   :2022-01-01 00:01:48.00   Length:4369360     Length:4369360    
##  1st Qu.:2022-05-29 10:54:29.50   Class :character   Class :character  
##  Median :2022-07-20 21:41:19.00   Mode  :character   Mode  :character  
##  Mean   :2022-07-19 14:24:52.86                                        
##  3rd Qu.:2022-09-14 18:39:00.00                                        
##  Max.   :2023-01-01 18:09:37.00                                        
##                                                                        
##  end_station_name   end_station_id       start_lat       start_lng     
##  Length:4369360     Length:4369360     Min.   :41.65   Min.   :-87.83  
##  Class :character   Class :character   1st Qu.:41.88   1st Qu.:-87.66  
##  Mode  :character   Mode  :character   Median :41.90   Median :-87.64  
##                                        Mean   :41.90   Mean   :-87.64  
##                                        3rd Qu.:41.93   3rd Qu.:-87.63  
##                                        Max.   :45.64   Max.   :-73.80  
##                                                                        
##     end_lat         end_lng       member_casual      ride_duration    
##  Min.   : 0.00   Min.   :-87.83   Length:4369360     Length:4369360   
##  1st Qu.:41.88   1st Qu.:-87.66   Class :character   Class :difftime  
##  Median :41.90   Median :-87.64   Mode  :character   Mode  :numeric   
##  Mean   :41.90   Mean   :-87.64                                       
##  3rd Qu.:41.93   3rd Qu.:-87.63                                       
##  Max.   :42.06   Max.   :  0.00                                       
##                                                                       
##      month         weekday        start_hour       route          
##  Jul    : 642680   Sun:599049   Min.   : 0.00   Length:4369360    
##  Jun    : 620350   Mon:585930   1st Qu.:11.00   Class :character  
##  Aug    : 605325   Tue:607639   Median :15.00   Mode  :character  
##  Sep    : 535145   Wed:616371   Mean   :14.21                     
##  May    : 502545   Thu:645897   3rd Qu.:18.00                     
##  Oct    : 414269   Fri:608851   Max.   :23.00                     
##  (Other):1049046   Sat:705623
\end{verbatim}

\begin{Shaded}
\begin{Highlighting}[]
\NormalTok{trips2021\_v2 }\OtherTok{\textless{}{-}}\NormalTok{ trips2021\_v2 }\SpecialCharTok{\%\textgreater{}\%} 
  \FunctionTok{select}\NormalTok{(member\_casual, rideable\_type, }\FunctionTok{ends\_with}\NormalTok{(}\StringTok{"name"}\NormalTok{),route, ride\_duration}\SpecialCharTok{:}\NormalTok{start\_hour) }\SpecialCharTok{\%\textgreater{}\%} 
  \FunctionTok{rename}\NormalTok{(}\AttributeTok{rider\_type =}\NormalTok{ member\_casual, }\AttributeTok{bike\_type =}\NormalTok{ rideable\_type)}
\end{Highlighting}
\end{Shaded}

\hypertarget{checking-for-zero-trip-duration}{%
\subsection{Checking for zero trip
duration}\label{checking-for-zero-trip-duration}}

\begin{Shaded}
\begin{Highlighting}[]
\NormalTok{trips2021\_v2 }\SpecialCharTok{\%\textgreater{}\%} 
  \FunctionTok{filter}\NormalTok{(ride\_duration }\SpecialCharTok{==} \DecValTok{0}\NormalTok{) }
\end{Highlighting}
\end{Shaded}

\begin{verbatim}
## # A tibble: 48,493 x 9
##    rider_type bike_type  start_station_name end_station_name route ride_duration
##    <chr>      <chr>      <chr>              <chr>            <chr> <drtn>       
##  1 member     classic_b~ State St & Pearso~ State St & Pear~ Stat~ 0 mins       
##  2 member     classic_b~ Clinton St & Lake~ Clinton St & La~ Clin~ 0 mins       
##  3 member     classic_b~ Lincoln Ave & Wav~ Lincoln Ave & W~ Linc~ 0 mins       
##  4 member     classic_b~ Financial Pl & Id~ Financial Pl & ~ Fina~ 0 mins       
##  5 member     classic_b~ Western Ave & 21s~ Western Ave & 2~ West~ 0 mins       
##  6 member     electric_~ Rush St & Superio~ Rush St & Super~ Rush~ 0 mins       
##  7 member     classic_b~ Halsted St & Will~ Halsted St & Wi~ Hals~ 0 mins       
##  8 member     classic_b~ Clinton St & Lake~ Clinton St & La~ Clin~ 0 mins       
##  9 member     classic_b~ Clarendon Ave & L~ Clarendon Ave &~ Clar~ 0 mins       
## 10 casual     electric_~ N Green St & W La~ N Green St & W ~ N Gr~ 0 mins       
## # i 48,483 more rows
## # i 3 more variables: month <ord>, weekday <ord>, start_hour <int>
\end{verbatim}

\hypertarget{checking-for-negative-trip-duration}{%
\subsection{Checking for negative trip
duration}\label{checking-for-negative-trip-duration}}

\begin{Shaded}
\begin{Highlighting}[]
\NormalTok{trips2021\_v2 }\SpecialCharTok{\%\textgreater{}\%} 
  \FunctionTok{filter}\NormalTok{(ride\_duration }\SpecialCharTok{\textless{}} \DecValTok{0}\NormalTok{)}
\end{Highlighting}
\end{Shaded}

\begin{verbatim}
## # A tibble: 56 x 9
##    rider_type bike_type  start_station_name end_station_name route ride_duration
##    <chr>      <chr>      <chr>              <chr>            <chr> <drtn>       
##  1 casual     classic_b~ DuSable Lake Shor~ DuSable Lake Sh~ DuSa~   -6 mins    
##  2 casual     electric_~ W Armitage Ave & ~ W Armitage Ave ~ W Ar~ -127 mins    
##  3 member     electric_~ Base - 2132 W Hub~ W Armitage Ave ~ Base~ -129 mins    
##  4 member     classic_b~ Lincoln Ave & Ros~ Lincoln Ave & R~ Linc~   -8 mins    
##  5 member     classic_b~ Lincoln Ave & Ros~ Lincoln Ave & R~ Linc~   -8 mins    
##  6 casual     classic_b~ Lincoln Ave & Ros~ Lincoln Ave & R~ Linc~  -21 mins    
##  7 casual     classic_b~ Lincoln Ave & Ros~ Lincoln Ave & R~ Linc~   -3 mins    
##  8 member     electric_~ Leavitt St & Chic~ Leavitt St & Ch~ Leav~   -1 mins    
##  9 casual     classic_b~ Lincoln Ave & Ros~ Lincoln Ave & R~ Linc~   -5 mins    
## 10 casual     classic_b~ Lincoln Ave & Ros~ Lincoln Ave & R~ Linc~   -8 mins    
## # i 46 more rows
## # i 3 more variables: month <ord>, weekday <ord>, start_hour <int>
\end{verbatim}

These rows will be removed for the sake of this analysis.

\hypertarget{remove-the-rows-where-the-ride-duration-is-either-zero-or-negative}{%
\subsection{Remove the rows where the ride duration is either zero or
negative}\label{remove-the-rows-where-the-ride-duration-is-either-zero-or-negative}}

\begin{Shaded}
\begin{Highlighting}[]
\NormalTok{trips2021\_cleaned }\OtherTok{\textless{}{-}}\NormalTok{ trips2021\_v2 }\SpecialCharTok{\%\textgreater{}\%} 
  \FunctionTok{filter}\NormalTok{(}\SpecialCharTok{!}\NormalTok{ride\_duration }\SpecialCharTok{\textless{}=} \DecValTok{0}\NormalTok{)}

\FunctionTok{head}\NormalTok{(trips2021\_cleaned)}
\end{Highlighting}
\end{Shaded}

\begin{verbatim}
## # A tibble: 6 x 9
##   rider_type bike_type   start_station_name end_station_name route ride_duration
##   <chr>      <chr>       <chr>              <chr>            <chr> <drtn>       
## 1 casual     electric_b~ Glenwood Ave & To~ Clark St & Touh~ Glen~  3 mins      
## 2 casual     electric_b~ Glenwood Ave & To~ Clark St & Touh~ Glen~  4 mins      
## 3 member     classic_bi~ Sheffield Ave & F~ Greenview Ave &~ Shef~  4 mins      
## 4 casual     classic_bi~ Clark St & Bryn M~ Paulina St & Mo~ Clar~ 15 mins      
## 5 member     classic_bi~ Michigan Ave & Ja~ State St & Rand~ Mich~  6 mins      
## 6 member     classic_bi~ Wood St & Chicago~ Honore St & Div~ Wood~  3 mins      
## # i 3 more variables: month <ord>, weekday <ord>, start_hour <int>
\end{verbatim}

\begin{Shaded}
\begin{Highlighting}[]
\FunctionTok{summary}\NormalTok{(trips2021\_cleaned)}
\end{Highlighting}
\end{Shaded}

\begin{verbatim}
##   rider_type         bike_type         start_station_name end_station_name  
##  Length:4320811     Length:4320811     Length:4320811     Length:4320811    
##  Class :character   Class :character   Class :character   Class :character  
##  Mode  :character   Mode  :character   Mode  :character   Mode  :character  
##                                                                             
##                                                                             
##                                                                             
##                                                                             
##     route           ride_duration         month         weekday     
##  Length:4320811     Length:4320811    Jul    : 635375   Sun:592220  
##  Class :character   Class :difftime   Jun    : 613662   Mon:579608  
##  Mode  :character   Mode  :numeric    Aug    : 598535   Tue:601066  
##                                       Sep    : 529057   Wed:609514  
##                                       May    : 497079   Thu:638854  
##                                       Oct    : 409211   Fri:602036  
##                                       (Other):1037892   Sat:697513  
##    start_hour   
##  Min.   : 0.00  
##  1st Qu.:11.00  
##  Median :15.00  
##  Mean   :14.21  
##  3rd Qu.:18.00  
##  Max.   :23.00  
## 
\end{verbatim}

\hypertarget{analyse}{%
\subsubsection{ANALYSE}\label{analyse}}

\hypertarget{riders-distribution}{%
\subsection{1. Riders distribution}\label{riders-distribution}}

\begin{Shaded}
\begin{Highlighting}[]
\NormalTok{trips2021\_cleaned }\SpecialCharTok{\%\textgreater{}\%} 
  \FunctionTok{group\_by}\NormalTok{(rider\_type) }\SpecialCharTok{\%\textgreater{}\%} 
  \FunctionTok{tally}\NormalTok{() }\SpecialCharTok{\%\textgreater{}\%} 
  \FunctionTok{mutate}\NormalTok{(}\AttributeTok{percentage =} \FunctionTok{round}\NormalTok{(n}\SpecialCharTok{/}\FunctionTok{sum}\NormalTok{(n)}\SpecialCharTok{*}\DecValTok{100}\NormalTok{)) }\SpecialCharTok{\%\textgreater{}\%} 
  \FunctionTok{ggplot}\NormalTok{(}\FunctionTok{aes}\NormalTok{(}\AttributeTok{x =} \DecValTok{1}\NormalTok{, }\AttributeTok{y =}\NormalTok{ percentage, }\AttributeTok{fill =}\NormalTok{ rider\_type)) }\SpecialCharTok{+}
  \FunctionTok{geom\_bar}\NormalTok{(}\AttributeTok{stat =} \StringTok{"identity"}\NormalTok{, }\AttributeTok{width =} \DecValTok{1}\NormalTok{) }\SpecialCharTok{+}
  \FunctionTok{geom\_text}\NormalTok{(}\FunctionTok{aes}\NormalTok{(}\AttributeTok{label =} \FunctionTok{str\_c}\NormalTok{(rider\_type, }\FunctionTok{str\_c}\NormalTok{(percentage, }\StringTok{"\%"}\NormalTok{), }\AttributeTok{sep =} \StringTok{"}\SpecialCharTok{\textbackslash{}n}\StringTok{"}\NormalTok{)), }
            \AttributeTok{position =} \FunctionTok{position\_stack}\NormalTok{(}\AttributeTok{vjust =} \FloatTok{0.5}\NormalTok{), }\AttributeTok{color =} \StringTok{"white"}\NormalTok{, }\AttributeTok{size =} \DecValTok{8}\NormalTok{) }\SpecialCharTok{+} 
  \FunctionTok{labs}\NormalTok{(}\AttributeTok{title =} \StringTok{"Riders Distribution"}\NormalTok{, }\AttributeTok{fill =} \StringTok{"Rider type"}\NormalTok{) }\SpecialCharTok{+} 
  \FunctionTok{coord\_polar}\NormalTok{(}\AttributeTok{theta =} \StringTok{"y"}\NormalTok{) }\SpecialCharTok{+}
  \FunctionTok{theme\_void}\NormalTok{()}
\end{Highlighting}
\end{Shaded}

\includegraphics{cycle_files/figure-latex/Riders distribution-1.pdf}
Members use Cylistic bikes more than the casual riders.Members account
for 55\% of the total rides while casual riders completed 45\% of the
total rides.Let's examine the average duration of the trips \#\#
2.Average ride duration

\begin{Shaded}
\begin{Highlighting}[]
\NormalTok{trips2021\_cleaned }\SpecialCharTok{\%\textgreater{}\%} 
  \FunctionTok{group\_by}\NormalTok{(rider\_type) }\SpecialCharTok{\%\textgreater{}\%} 
  \FunctionTok{summarise}\NormalTok{(}\AttributeTok{avg\_ride\_duration =} \FunctionTok{round}\NormalTok{(}\FunctionTok{mean}\NormalTok{(ride\_duration)))}\SpecialCharTok{\%\textgreater{}\%} 
  \FunctionTok{ggplot}\NormalTok{(}\FunctionTok{aes}\NormalTok{(}\AttributeTok{x =}\NormalTok{ rider\_type, }\AttributeTok{y =}\NormalTok{ avg\_ride\_duration)) }\SpecialCharTok{+} 
  \FunctionTok{geom\_col}\NormalTok{(}\AttributeTok{position =} \StringTok{"dodge"}\NormalTok{, }\AttributeTok{fill =} \StringTok{"black"}\NormalTok{) }\SpecialCharTok{+}
  \FunctionTok{labs}\NormalTok{( }\AttributeTok{x =} \StringTok{"Rider type"}\NormalTok{, }\AttributeTok{y =} \StringTok{"Average ride duration (mins)"}\NormalTok{)}
\end{Highlighting}
\end{Shaded}

\begin{verbatim}
## Don't know how to automatically pick scale for object of type <difftime>.
## Defaulting to continuous.
\end{verbatim}

\includegraphics{cycle_files/figure-latex/Average ride duration-1.pdf}

casual members ride the bikes longer than members. The average ride
duration for casual riders is more than twice for members.

\hypertarget{compute-and-visualize-the-monthly-ride-distribution}{%
\subsection{3.Compute and visualize the monthly ride
distribution}\label{compute-and-visualize-the-monthly-ride-distribution}}

\begin{Shaded}
\begin{Highlighting}[]
\NormalTok{trips2021\_cleaned }\SpecialCharTok{\%\textgreater{}\%} 
  \FunctionTok{group\_by}\NormalTok{(rider\_type, month) }\SpecialCharTok{\%\textgreater{}\%} 
  \FunctionTok{summarise}\NormalTok{(}\AttributeTok{total\_rides =} \FunctionTok{n}\NormalTok{()) }\SpecialCharTok{\%\textgreater{}\%} 
  \FunctionTok{arrange}\NormalTok{(month) }\SpecialCharTok{\%\textgreater{}\%} 
  \FunctionTok{ggplot}\NormalTok{(}\FunctionTok{aes}\NormalTok{(}\AttributeTok{x =}\NormalTok{ month, }\AttributeTok{y =}\NormalTok{ total\_rides, }\AttributeTok{fill =}\NormalTok{ rider\_type)) }\SpecialCharTok{+}
  \FunctionTok{geom\_col}\NormalTok{(}\AttributeTok{position =} \StringTok{"dodge"}\NormalTok{) }\SpecialCharTok{+}
  \FunctionTok{labs}\NormalTok{(}\AttributeTok{title =} \StringTok{"Monthly bike rides by rider type"}\NormalTok{,}
       \AttributeTok{x =} \StringTok{"Month"}\NormalTok{, }\AttributeTok{y =} \StringTok{"Number of rides"}\NormalTok{, }\AttributeTok{fill =} \StringTok{"Rider type"}\NormalTok{)}
\end{Highlighting}
\end{Shaded}

\begin{verbatim}
## `summarise()` has grouped output by 'rider_type'. You can override using the
## `.groups` argument.
\end{verbatim}

\includegraphics{cycle_files/figure-latex/Compute and visualize the monthly ride distribution-1.pdf}
The casual riders' bike usage was significantly lower than the usage by
members from February to April. The bike usage by casual riders and
members started to rise in the spring (from March to May) following a
dip in the winter months (December to March) with members leading the
pack.

\hypertarget{distribution-of-weekly-bike-usage}{%
\subsection{4.Distribution of weekly bike
usage}\label{distribution-of-weekly-bike-usage}}

\begin{Shaded}
\begin{Highlighting}[]
\NormalTok{trips2021\_cleaned }\SpecialCharTok{\%\textgreater{}\%} 
  \FunctionTok{group\_by}\NormalTok{(rider\_type, weekday) }\SpecialCharTok{\%\textgreater{}\%} 
  \FunctionTok{summarise}\NormalTok{(}\AttributeTok{total\_rides =} \FunctionTok{n}\NormalTok{()) }\SpecialCharTok{\%\textgreater{}\%} 
  \FunctionTok{ggplot}\NormalTok{(}\FunctionTok{aes}\NormalTok{(}\AttributeTok{x =}\NormalTok{ weekday, }\AttributeTok{y =}\NormalTok{ total\_rides, }\AttributeTok{fill =}\NormalTok{ rider\_type)) }\SpecialCharTok{+}
  \FunctionTok{geom\_col}\NormalTok{(}\AttributeTok{position =} \StringTok{"dodge"}\NormalTok{) }\SpecialCharTok{+}
  \FunctionTok{labs}\NormalTok{(}\AttributeTok{title =} \StringTok{"Weekly bike usage distribution"}\NormalTok{, }
       \AttributeTok{fill =} \StringTok{"Rider type"}\NormalTok{, }\AttributeTok{x =} \StringTok{"Weekday"}\NormalTok{,}
       \AttributeTok{y =} \StringTok{"Number of rides"}\NormalTok{)}
\end{Highlighting}
\end{Shaded}

\begin{verbatim}
## `summarise()` has grouped output by 'rider_type'. You can override using the
## `.groups` argument.
\end{verbatim}

\includegraphics{cycle_files/figure-latex/Distribution of weekly bike usage-1.pdf}
We observed that casual riders seem to use the bikes more for leisure
while the members seem more likely to use the bike to commute to and
from work. Casual riders used the bikes far more on weekends. Their
usage starts to rise on Fridays and moves up significantly on Saturdays
and Sundays from the fairly consistent level on weekdays. Members' usage
is fairly consistent throughout the week. \#\# 5.Hourly Distribution of
bike Usage

\begin{Shaded}
\begin{Highlighting}[]
\NormalTok{trips2021\_cleaned }\SpecialCharTok{\%\textgreater{}\%} 
  \FunctionTok{group\_by}\NormalTok{(rider\_type, start\_hour) }\SpecialCharTok{\%\textgreater{}\%} 
  \FunctionTok{summarise}\NormalTok{(}\AttributeTok{total\_rides =} \FunctionTok{n}\NormalTok{()) }\SpecialCharTok{\%\textgreater{}\%} 
  \FunctionTok{ggplot}\NormalTok{(}\FunctionTok{aes}\NormalTok{(}\AttributeTok{x =}\NormalTok{ start\_hour, }\AttributeTok{y =}\NormalTok{ total\_rides, }\AttributeTok{fill =}\NormalTok{ rider\_type)) }\SpecialCharTok{+}
  \FunctionTok{geom\_col}\NormalTok{(}\AttributeTok{position =} \StringTok{"dodge"}\NormalTok{) }\SpecialCharTok{+}
    \FunctionTok{scale\_x\_continuous}\NormalTok{(}\AttributeTok{breaks =} \FunctionTok{c}\NormalTok{(}\DecValTok{0}\SpecialCharTok{:}\DecValTok{23}\NormalTok{))}\SpecialCharTok{+}
  \FunctionTok{labs}\NormalTok{(}\AttributeTok{title =} \StringTok{"Hourly bike rides"}\NormalTok{, }
       \AttributeTok{fill =} \StringTok{"Rider type"}\NormalTok{, }\AttributeTok{x =} \StringTok{"Hour"}\NormalTok{,}
       \AttributeTok{y =} \StringTok{"Number of rides"}\NormalTok{)}
\end{Highlighting}
\end{Shaded}

\begin{verbatim}
## `summarise()` has grouped output by 'rider_type'. You can override using the
## `.groups` argument.
\end{verbatim}

\includegraphics{cycle_files/figure-latex/Hourly Distribution of bike Usage-1.pdf}
Members use the bikes significantly more than the casual riders from 6
a.m to 9 a.m in the morning and between 4 p.m to 7 p.m in the evening.
These pattern agrees with our hypothesis that the members use the bikes
more for work.

\hypertarget{most-poular-stations-where-the-riders-start-their-trip-from}{%
\subsection{6.Most poular stations where the riders start their trip
from}\label{most-poular-stations-where-the-riders-start-their-trip-from}}

\begin{Shaded}
\begin{Highlighting}[]
\NormalTok{trips2021\_cleaned }\SpecialCharTok{\%\textgreater{}\%} 
  \FunctionTok{group\_by}\NormalTok{(rider\_type, start\_station\_name) }\SpecialCharTok{\%\textgreater{}\%} 
  \FunctionTok{filter}\NormalTok{(rider\_type }\SpecialCharTok{==} \StringTok{"member"}\NormalTok{) }\SpecialCharTok{\%\textgreater{}\%} 
  \FunctionTok{summarise}\NormalTok{(}\AttributeTok{Number\_of\_rides =} \FunctionTok{n}\NormalTok{()) }\SpecialCharTok{\%\textgreater{}\%} 
  \FunctionTok{arrange}\NormalTok{(}\FunctionTok{desc}\NormalTok{(Number\_of\_rides)) }\SpecialCharTok{\%\textgreater{}\%} 
  \FunctionTok{head}\NormalTok{(}\DecValTok{10}\NormalTok{) }\SpecialCharTok{\%\textgreater{}\%} 
   \FunctionTok{ggplot}\NormalTok{(}\FunctionTok{aes}\NormalTok{(}\AttributeTok{x =}\NormalTok{ Number\_of\_rides, }\AttributeTok{y =} \FunctionTok{reorder}\NormalTok{(start\_station\_name, Number\_of\_rides))) }\SpecialCharTok{+} 
  \FunctionTok{geom\_col}\NormalTok{() }\SpecialCharTok{+}
  \FunctionTok{labs}\NormalTok{(}\AttributeTok{title =} \StringTok{"Most popular start station for members"}\NormalTok{,}
       \AttributeTok{x =} \StringTok{"Number of rides"}\NormalTok{, }\AttributeTok{y =} \StringTok{"Start station name"}\NormalTok{)}
\end{Highlighting}
\end{Shaded}

\begin{verbatim}
## `summarise()` has grouped output by 'rider_type'. You can override using the
## `.groups` argument.
\end{verbatim}

\includegraphics{cycle_files/figure-latex/Top 10 Most poular stations for members-1.pdf}

\begin{Shaded}
\begin{Highlighting}[]
\NormalTok{trips2021\_cleaned }\SpecialCharTok{\%\textgreater{}\%} 
  \FunctionTok{group\_by}\NormalTok{(rider\_type, start\_station\_name) }\SpecialCharTok{\%\textgreater{}\%} 
  \FunctionTok{filter}\NormalTok{(rider\_type }\SpecialCharTok{==} \StringTok{"casual"}\NormalTok{) }\SpecialCharTok{\%\textgreater{}\%} 
  \FunctionTok{summarise}\NormalTok{(}\AttributeTok{Number\_of\_rides =} \FunctionTok{n}\NormalTok{()) }\SpecialCharTok{\%\textgreater{}\%} 
  \FunctionTok{arrange}\NormalTok{(}\FunctionTok{desc}\NormalTok{(Number\_of\_rides)) }\SpecialCharTok{\%\textgreater{}\%} 
  \FunctionTok{head}\NormalTok{(}\DecValTok{10}\NormalTok{) }\SpecialCharTok{\%\textgreater{}\%} 
  \FunctionTok{ggplot}\NormalTok{(}\FunctionTok{aes}\NormalTok{(}\AttributeTok{x =}\NormalTok{ Number\_of\_rides, }\AttributeTok{y =} \FunctionTok{reorder}\NormalTok{(start\_station\_name, Number\_of\_rides))) }\SpecialCharTok{+} 
  \FunctionTok{geom\_col}\NormalTok{() }\SpecialCharTok{+}
  \FunctionTok{labs}\NormalTok{(}\AttributeTok{title =} \StringTok{"Most popular start station for casual riders"}\NormalTok{,}
       \AttributeTok{x =} \StringTok{"Number of rides"}\NormalTok{, }\AttributeTok{y =} \StringTok{"Start station name"}\NormalTok{)}
\end{Highlighting}
\end{Shaded}

\begin{verbatim}
## `summarise()` has grouped output by 'rider_type'. You can override using the
## `.groups` argument.
\end{verbatim}

\includegraphics{cycle_files/figure-latex/Top 10 Most poular stations for casual riders-1.pdf}

The top start stations are different for members and casual riders.
Streeter Dr \& Grand Ave is by far the most popular station for casual
riders followed by Millennium Park and Michigan Ave and Oak St.~The top
three start stations for members are Clark St \& Elm St, Wells St \&
Concord Ln, and Kingsbury St \& Kinzie St.

\hypertarget{most-popular-route-for-members-with-average-duration}{%
\subsection{7.Most popular route for members with average
duration}\label{most-popular-route-for-members-with-average-duration}}

\begin{Shaded}
\begin{Highlighting}[]
\CommentTok{\#For Members}
\CommentTok{\#trips2021\_cleaned \%\textgreater{}\% }
 \CommentTok{\# group\_by(rider\_type, route) \%\textgreater{}\% }
 \CommentTok{\# summarise(Number\_of\_rides = n()) \%\textgreater{}\% }
 \CommentTok{\# arrange(desc(Number\_of\_rides)) \%\textgreater{}\% }
 \CommentTok{\# head(10)}

\CommentTok{\#route\_member \textless{}{-} trips2021\_cleaned \%\textgreater{}\% }
 \CommentTok{\# group\_by(rider\_type, route) \%\textgreater{}\% }
 \CommentTok{\# filter(rider\_type == "member") \%\textgreater{}\% }
 \CommentTok{\# summarise(Number\_of\_rides = n(), ride\_duration = mean(ride\_duration)) \%\textgreater{}\% }
 \CommentTok{\# arrange(desc(Number\_of\_rides)) \%\textgreater{}\% }
 \CommentTok{\# head(10)}
  \CommentTok{\#Plot of Most popular routes for members}
\CommentTok{\#ggplot(route\_member) + }
 \CommentTok{\#geom\_col(aes(x = Number\_of\_rides, y = reorder(route, Number\_of\_rides))) +}
  \CommentTok{\#labs(title = "Most popular route for members",}
    \CommentTok{\#  x = "Number of rides", y = "Route")}
\end{Highlighting}
\end{Shaded}

\begin{Shaded}
\begin{Highlighting}[]
\CommentTok{\#route\_casual \textless{}{-} trips2021\_cleaned \%\textgreater{}\% }
  \CommentTok{\#group\_by(rider\_type, route) \%\textgreater{}\% }
  \CommentTok{\#filter(rider\_type == "casual") \%\textgreater{}\% }
  \CommentTok{\#summarise(Number\_of\_rides = n(), avg\_ride\_duration = mean(ride\_duration)) \%\textgreater{}\% }
  \CommentTok{\#arrange(desc(Number\_of\_rides)) \%\textgreater{}\% }
  \CommentTok{\#head(10)}

 \CommentTok{\# Plot of Most popular routes for casual riders}
\CommentTok{\#ggplot(route\_casual) + }
 \CommentTok{\# geom\_col(aes(x = Number\_of\_rides, y = reorder(route, Number\_of\_rides))) +}
  \CommentTok{\#labs(title = "Most Popular route for Casual Riders",}
     \CommentTok{\#  x = "Number of rides", y = "Route")}
\end{Highlighting}
\end{Shaded}

\hypertarget{types-of-bike-used-by-the-riders}{%
\subsection{8.Types of bike used by the
riders}\label{types-of-bike-used-by-the-riders}}

\begin{Shaded}
\begin{Highlighting}[]
\NormalTok{trips2021\_cleaned }\SpecialCharTok{\%\textgreater{}\%} 
  \FunctionTok{group\_by}\NormalTok{(rider\_type, bike\_type) }\SpecialCharTok{\%\textgreater{}\%} 
  \FunctionTok{summarise}\NormalTok{(}\AttributeTok{Number\_of\_bikes =} \FunctionTok{n}\NormalTok{()) }\SpecialCharTok{\%\textgreater{}\%} 
  \FunctionTok{ggplot}\NormalTok{(}\FunctionTok{aes}\NormalTok{(}\AttributeTok{x =}\NormalTok{ bike\_type, }\AttributeTok{y =}\NormalTok{ Number\_of\_bikes, }\AttributeTok{fill =}\NormalTok{ rider\_type)) }\SpecialCharTok{+} 
  \FunctionTok{geom\_col}\NormalTok{(}\AttributeTok{position =} \StringTok{"dodge"}\NormalTok{) }\SpecialCharTok{+}
  \FunctionTok{labs}\NormalTok{(}\AttributeTok{x =} \StringTok{"Bike type"}\NormalTok{, }\AttributeTok{y =} \StringTok{"Number of bikes"}\NormalTok{, }\AttributeTok{fill =} \StringTok{"Rider type"}\NormalTok{)}
\end{Highlighting}
\end{Shaded}

\begin{verbatim}
## `summarise()` has grouped output by 'rider_type'. You can override using the
## `.groups` argument.
\end{verbatim}

\includegraphics{cycle_files/figure-latex/Types of bike used by the riders-1.pdf}

\hypertarget{share}{%
\subsubsection{Share}\label{share}}

The key insights gleaned from the analysis are as follows:

\begin{itemize}
\item
  Members use Cyclist bikes more than the casual riders. Members account
  for 55\% of the total rides while casual riders completed 45\% of the
  total rides. However, casual riders ride the bikes longer than
  members. The average ride duration for casual riders is more than
  twice for members.
\item
  Casual riders use the Cyclist bikes more than members in the summer
  months, July to September. Members use the bikes more than casual
  riders in the remaining months of the year (outside summer months).
\item
  Casual riders seem to use the bikes more for leisure while the members
  seem more likely to use the bike to commute to and from work. Casual
  riders used the bikes far more on weekends. Their usage starts to rise
  on Fridays and moves up significantly on Saturdays and Sundays from
  the fairly consistent level on weekdays. Members' usage is fairly
  consistent throughout the week.
\item
  The bikes are mostly used during the day by both categories of users.
  Members use the bikes significantly more than the casual riders from 6
  a.m to 9 a.m in the morning and between 4 p.m to 7 p.m in the evening.
  Casual riders use the bikes more than members in the night from 9pm
  and at odd hours.
\item
  The top 2 most popular stations where casual riders start their trips
  from are close to leisure centres. Pointing us to the conclusion that
  they use Cyclist bikes primarily for leisure.
\item
  Casual members generally ride the bikes longer than members throughout
  the hours of the day. The average ride duration is fairly constant for
  members throughout the day. It is interesting to note that despite
  that fewer rides happen at odd hours from 12 midnight to 4 a.m, casual
  riders ride the bikes for longer duration during those periods.
\end{itemize}

\end{document}
